\chapter{Fermilab Resources}

Fermilab provides several resources for use by Muon $g-2$ collaborators. A description and usage information are described here.

\section{\index{GPCF}\index{gm2gpvm}General Physics Computing Facility (GPCF)}

The General Physics Computing Facility (GPCF) is a farm of virtualized interactive login machines located at Fermilab. These machines are meant for interactive use (compiling code, running \texttt{root}, etc). These machines are also a gateway to other Fermilab resources (storage, Fermigrid, etc).

\subsection{Quick Start FAQ}

\index{gm2gpvm!accessing}
\begin{description}[style=nextline]
\item[Q: How do I get an account?] A: If you cannot log in (see below), then see the \href{https://cdcvs.fnal.gov/redmine/projects/g-2/wiki/NewGm2Person}{Welcome Packet}.\\

\item[Q: Where do I log in?] A: Get a forwardable kerberos ticket (\texttt{kinit -f}) and \texttt{ssh} to \texttt{gm2gpvm.fnal.gov}. \textbf{IMPORTANT} You must configure your ssh client correctly. See \ref{sec:configssh}\fxnote{Write configssh section}.\\

\item[Q: How come when I ssh to gm2gpvm I get a machine called gm2gpvm01 (or 02, 03)?] A: We have several identical virtual machines for g-2. In order to spread the interactive load, we have a special "DNS Endpoint" called \textsf{gm2gpvm} which chooses a particular machine for you in a round-robin fashion. See \ref{sc:configssh} for more information. \\

\item[Q: Where can I put stuff?] A: In your home area. If you cannot write to your home area (you get permission denied), do a \texttt{kinit}, type in your \textit{kerberos} password, and things will work.\\

\item[Q: Where else can I put stuff?] A: If you are writing applications, you can make a directory at \texttt{/gm2/app/users}. See below for instructions.\\

\item[Q: What can I run?] A: See \ref{sec:software}\fxnote{Write software section}.\\

\item[Q: How can I run VNC?] A: See section \ref{sec:vnc}\fxnote{Write VNC section} for VNC information.

\end{description}

\subsection{General information}

Muon $g-2$ currently has three virtual machines running Scientific Linux 5 (SL5) on GPCF called \texttt{gm2gpvm01}, \texttt{gm2gpvm02}, and \texttt{gm2gpvm03}. There is also another VM running SL6 called \texttt{gm2gpvm04}. WIth a kerberos ticket, you can ssh to one of these nodes and do work. So that you don't have to choose a particular machine, log into \texttt{gm2gpvm.fnal.gov} and one will be chosen for you (the SL6 node, \texttt{gm2gpvm04} is excluded). Be sure to read the instructions in section \ref{sec:configssh} to configure your ssh client correctly.

Your home area is your \texttt{afs} home directory. Since that area is backed up, you should put your important files and code there. There are other disk areas that are not backed up which are available. 

\begin{description}[style=nextline]

\item[\texttt{/gm2/app}]The main application area. Individuals can have space at \texttt{/gm2/app/users}. Simple \texttt{cd} there and make a directory with your name (e.g. \texttt{cd /gm2/app/users ; mkdir lyon}. If you plan to use a large amount of space, please send mail to \url{gm2-computing@fnal.gov} first. \\

\item[\texttt{/gm2/data}]The main data area. You can use this area to write ntuples and histograms. There is a similar \texttt{/gm2/data/users} structure to \texttt{/gm2/app}. 
\end{description}
  
